% Options for packages loaded elsewhere
\PassOptionsToPackage{unicode}{hyperref}
\PassOptionsToPackage{hyphens}{url}
\PassOptionsToPackage{dvipsnames,svgnames*,x11names*}{xcolor}
%
\documentclass[
  12pt,
]{krantz}
\usepackage{lmodern}
\usepackage{amssymb,amsmath}
\usepackage{ifxetex,ifluatex}
\ifnum 0\ifxetex 1\fi\ifluatex 1\fi=0 % if pdftex
  \usepackage[T1]{fontenc}
  \usepackage[utf8]{inputenc}
  \usepackage{textcomp} % provide euro and other symbols
\else % if luatex or xetex
  \usepackage{unicode-math}
  \defaultfontfeatures{Scale=MatchLowercase}
  \defaultfontfeatures[\rmfamily]{Ligatures=TeX,Scale=1}
\fi
% Use upquote if available, for straight quotes in verbatim environments
\IfFileExists{upquote.sty}{\usepackage{upquote}}{}
\IfFileExists{microtype.sty}{% use microtype if available
  \usepackage[]{microtype}
  \UseMicrotypeSet[protrusion]{basicmath} % disable protrusion for tt fonts
}{}
\makeatletter
\@ifundefined{KOMAClassName}{% if non-KOMA class
  \IfFileExists{parskip.sty}{%
    \usepackage{parskip}
  }{% else
    \setlength{\parindent}{0pt}
    \setlength{\parskip}{6pt plus 2pt minus 1pt}}
}{% if KOMA class
  \KOMAoptions{parskip=half}}
\makeatother
\usepackage{xcolor}
\IfFileExists{xurl.sty}{\usepackage{xurl}}{} % add URL line breaks if available
\IfFileExists{bookmark.sty}{\usepackage{bookmark}}{\usepackage{hyperref}}
\hypersetup{
  pdftitle={Pot pourri},
  colorlinks=true,
  linkcolor=Maroon,
  filecolor=Maroon,
  citecolor=Blue,
  urlcolor=Blue,
  pdfcreator={LaTeX via pandoc}}
\urlstyle{same} % disable monospaced font for URLs
\usepackage{longtable,booktabs}
% Correct order of tables after \paragraph or \subparagraph
\usepackage{etoolbox}
\makeatletter
\patchcmd\longtable{\par}{\if@noskipsec\mbox{}\fi\par}{}{}
\makeatother
% Allow footnotes in longtable head/foot
\IfFileExists{footnotehyper.sty}{\usepackage{footnotehyper}}{\usepackage{footnote}}
\makesavenoteenv{longtable}
\usepackage{graphicx,grffile}
\makeatletter
\def\maxwidth{\ifdim\Gin@nat@width>\linewidth\linewidth\else\Gin@nat@width\fi}
\def\maxheight{\ifdim\Gin@nat@height>\textheight\textheight\else\Gin@nat@height\fi}
\makeatother
% Scale images if necessary, so that they will not overflow the page
% margins by default, and it is still possible to overwrite the defaults
% using explicit options in \includegraphics[width, height, ...]{}
\setkeys{Gin}{width=\maxwidth,height=\maxheight,keepaspectratio}
% Set default figure placement to htbp
\makeatletter
\def\fps@figure{htbp}
\makeatother
\setlength{\emergencystretch}{3em} % prevent overfull lines
\providecommand{\tightlist}{%
  \setlength{\itemsep}{0pt}\setlength{\parskip}{0pt}}
\setcounter{secnumdepth}{5}
\usepackage{booktabs}
\usepackage{longtable}
\usepackage{layouts}
\usepackage[bf,singlelinecheck=off]{caption}


\usepackage{framed,color}
\definecolor{shadecolor}{RGB}{248,248,248}

\usepackage{float}
\floatplacement{figure}{H}
\usepackage{makeidx}
\makeindex

% The following commands make floating environments less likely 
% to float by allowing them to occupy larger fractions of pages 
% without floating.
\renewcommand{\textfraction}{0.05}
\renewcommand{\topfraction}{0.8}
\renewcommand{\bottomfraction}{0.8}
\renewcommand{\floatpagefraction}{0.75}

%Since krantz.cls provided an environment VF for quotes, we redefine the standard quote environment to VF. You can see its style in Section 2.1.

\renewenvironment{quote}{\begin{VF}}{\end{VF}}



\makeatletter
\newenvironment{kframe}{%
\medskip{}
\setlength{\fboxsep}{.8em}
 \def\at@end@of@kframe{}%
 \ifinner\ifhmode%
  \def\at@end@of@kframe{\end{minipage}}%
  \begin{minipage}{\columnwidth}%
 \fi\fi%
 \def\FrameCommand##1{\hskip\@totalleftmargin \hskip-\fboxsep
 \colorbox{shadecolor}{##1}\hskip-\fboxsep
     % There is no \\@totalrightmargin, so:
     \hskip-\linewidth \hskip-\@totalleftmargin \hskip\columnwidth}%
 \MakeFramed {\advance\hsize-\width
   \@totalleftmargin\z@ \linewidth\hsize
   \@setminipage}}%
 {\par\unskip\endMakeFramed%
 \at@end@of@kframe}
\makeatother

\makeatletter
\@ifundefined{Shaded}{
}{\renewenvironment{Shaded}{\begin{kframe}}{\end{kframe}}}
\makeatother

\newenvironment{rmdblock}[1]
  {
  \begin{itemize}
  \renewcommand{\labelitemi}{
    \raisebox{-.7\height}[0pt][0pt]{
      {\setkeys{Gin}{width=3em,keepaspectratio}\includegraphics{img/#1}}
    }
  }
  \setlength{\fboxsep}{1em}
  \begin{kframe}
  \item
  }
  {
  \end{kframe}
  \end{itemize}
  }
\newenvironment{rmdnote}
  {\begin{rmdblock}{note}}
  {\end{rmdblock}}
\newenvironment{rmdcaution}
  {\begin{rmdblock}{caution}}
  {\end{rmdblock}}
\newenvironment{rmdimportant}
  {\begin{rmdblock}{important}}
  {\end{rmdblock}}
\newenvironment{rmdtip}
  {\begin{rmdblock}{tip}}
  {\end{rmdblock}}
\newenvironment{rmdwarning}
  {\begin{rmdblock}{warning}}
  {\end{rmdblock}}
\newenvironment{rmdquestion}
  {\begin{rmdblock}{question}}
  {\end{rmdblock}}
\newenvironment{rmdyou}
  {\begin{rmdblock}{you}}
  {\end{rmdblock}}
\newenvironment{rmdobjective}
  {\begin{rmdblock}{objective}}
  {\end{rmdblock}}
\newenvironment{rmdinfo}
  {\begin{rmdblock}{caution}}
  {\end{rmdblock}}
\newenvironment{rmdbible}
  {\begin{rmdblock}{bible}}
  {\end{rmdblock}}
\newenvironment{rmdquote}
  {\begin{rmdblock}{quote}}
  {\end{rmdblock}}
\newenvironment{rmddefinition}
  {\begin{rmdblock}{definition}}
  {\end{rmdblock}}

%Then we redefine hyperlinks to be footnotes, because when the book is printed on paper, readers are not able to click on links in text. Footnotes will tell them what the actual links are.

\let\oldhref\href
\renewcommand{\href}[2]{#2\footnote{\url{#1}}}




\frontmatter
\usepackage[]{natbib}
\bibliographystyle{apalike}

\title{Pot pourri}
\author{}
\date{\vspace{-2.5em}2020-10-18}

\begin{document}
\maketitle

\thispagestyle{empty}
\mainmatter

{
\hypersetup{linkcolor=}
\setcounter{tocdepth}{2}
\tableofcontents
}
\hypertarget{uxfcbersicht}{%
\chapter{Übersicht}\label{uxfcbersicht}}

\begin{itemize}
\tightlist
\item
  \protect\hyperlink{auf-festen-grund-gebaut}{Auf festen Grund gebaut - Auszüge}
\end{itemize}

\begin{rmdinfo}
\textbf{Soli Deo Gloria}
\end{rmdinfo}

\begin{center}\rule{0.5\linewidth}{0.5pt}\end{center}

Nur für private Zwecke.

\hypertarget{auf-festen-grund-gebaut}{%
\chapter{Auf festen Grund gebaut}\label{auf-festen-grund-gebaut}}

Auszüge aus der Einleitung von \href{https://www.cbuch.de/mcilwain-auf-festen-grund-gebaut.html}{``Auf festen Grund gebaut''}

\textbf{Welches ist die klarste, einfachste und dennoch verständlichste Methode
Gottes Wort zu lehren, um Menschen auf das Evangelium und Gottes Weg der
Erlösung vorzubereiten?}

{[}\ldots{]}

Christus und Seine freimachende Botschaft sind das einzige Fundament, das
Gott als Grundlage des Glaubens schuldiger Sünder bestimmt hat. {[}\ldots{]} Beim Bau
von Gebäuden muss zuerst das Fundament vorbereitet werden. Leider verläuft
ein Grossteil der Evangeliums-Verkündigung ohne fundierte Vorbereitungen. {[}\ldots{]}
Ein weiterer Fehler in der christliche Fortbildung ist das Versäumis, die
gesamte Bibel als eine Einheit zu präsentieren und zu lehren, so wie Gott
sie uns durch eine schrittweise enthüllende Offenbarung aufzeichnen liess.

{[}\ldots{]}

Nachdem sie jahrelang themenbezogene, eigenständige Predigten hörten, von denen
sich die meisten auf Verse, die aus dem Kontext herausgenommen wurden,
stützen, kennen viele Gemeindeglieder die Bibel immer noch nicht als ein
zusammenhängendes Buch. {[}\ldots{]} Viele Missionare brennen so sehr darauf, die
Frohe Botschaft zu verkündigen, dass sie es als Zeitverschwendung empfinden,
den Stammesleuten alttestamentliche Berichte vorzustellen. Trotzdem formen
gerade diese alttestamentlichen Abschnitte das gesunde Verständnis des Kommens
Christi, die Notwendigkeit Seines Todes, Seiner Grablegung und Seiner
Auferstehung. Werden die alttestamentlichen Schriften richtig gelehrt,
so bereiten sie das Herz des Sünders zur Aufnahme des Evangeliums in echter
Busse und auf den lebendigen Glauben vor.

\textbf{Was ist das Evangelium?}

Das Evangelium bezigeht sich in erster Linie und hauptsächlich auf Christus.
Es ist die Botschaft über das in Christus vollendete, historische Werk Gottes.
Das Evangelium ist einzig und allein das Werk der Gottheit. Christus war
``\emph{\ldots{} von Gott geschlagen\ldots{}}'', ``\emph{\ldots dem Herrn gefiel es, Ihn zu zerschlagen.
Er hat Ihn leiden lassen\ldots{}}'', der Herr hat ``\emph{\ldots Sein Leben als Schuldopfer
eingesetzt\ldots{}}'' (Jesaja 53.4-10).

Viele verwechseln das Evangelium, Gottes Werk durch Christus FÜR uns, mit
seinem Werk IN uns, bewirkt durch den Heiligen Geist. Das Evangelium ist
lediglich eine objektive Tatsache, die ausserhalb von uns geschehen ist. Diese
Botschaft spricht nicht von der Veränderung, die IN uns stattfinden muss und
es `passiert' auch nicht IN uns. Es wurde durch Christus vor etwa 2000 Jahren
vollbracht, völlig unabhängig von uns.

\textbf{Unbiblische Bezeichnungen}

Wir verdrehen und verzerren das Evangelium im Verständnis der Menschen, wenn
wir Worte in unserer Verkündigung verwenden, die die Aufmerksamkeit der Leute
darauf lenken, was zu TUN ist, anstatt zu verdeutlichen, was Gott für sie
in Christus GETAN hat, Wir sollen Begriffe verwenden, die den bussfertigen
Sündern zeigen, dass sie ihr Vertrauen auf das legen sollen, was durch Christus
FÜR SIE getan wurde, anstatt ihre Aufmerksamkeit auf das zu lenken, was IN
IHNEN zu tun ist.

\begin{quote}
Nimm Jesus in dein Herz auf.
Übergib Jesus dein Herz.
Gib dein Leben Jesus.
Öffne dem Herrn die Tür deines Herzens.
Bitte Jesus, deine Sünden abzuwaschen.
Entscheide dich für Christus.
Bitte Jesus, dir ewiges Leben zu schenken.
Bitte Gott dich zu erretten.
\end{quote}

Diese modernen und oft allgemein üblichen Phrasen verwirren das Verständnis
der Menschen für das Evangelium.

Wenn wir Menschen für das Evanglium vorbereiten, sollen wir sie zu dem Punkt
kommen lassen, an dem sie erkennen, dass \textbf{sie selbst nichts tun können}. Doch
selbst wenn Menschen erkennen, dass sie hilflos und unfähig sind irgendwas zu
tun, erzählen ihnen viele Evangelisten, Missionare und Verkündiger Dinge wie
folgende: ``Nun musst du dein Herz Jesu übergeben''. Soeben sagten sie den
Zuhörern, sie seien unfähig irgend etwas zu tun, nun forderten sie alle auf,
etwas zu tun. Mit welchem Ergebnis? Verwirrung in Bezug auf das Evangelium!

{[}\ldots{]}

Das Evangelium besteht nicht darin, dass der Mensch den Herrn Jesus als seinen
Heiland anerkennt, sondern, dass Gott vor 2000 Jahren den Herrn Jesus als den
einzigen Erlöser und Befreier anerkannte. Das Evangelium ist nicht der Mensch,
der sein Herz oder Leben Jesus gibt, sondern dass Christus Sein Leben, Seine
gesamte Existenz an Stelle der Sünders gegeben hat. Auch bedeutet Evangelium
nicht, dass der Mensch Christus in sein Herz aufnimmt, sondern, dass Gott
den Hernn Jesus als Mittler für Sünder in den Himmel aufgenommen hat. Zudem ist
das Evangelium auch nicht die Botschaft, dass Christus auf dem Thron des
menschlichen Herzens regiert, sondern das Gott den Herrn Jesus auf dem Thron
in der Himmelswelt zu Seiner Rechten sitzen liess.

Erkennen wir den grossen Unterschied zwischen diesen beiden Botschaften? Die
eine ist subjektiv und legt die Betonnung auf die Dinge, die der Mensch tun
muss. Die andere ist objektiv und legt den Schwerpunkt auf das, was Christus
schon getan hat. Der Sünder soll eigentlich nur auf das vertrauen, was um
seinetwillen bereits geschehen ist.

Es gibt einige, denen diese Art der Evangeliumsverkündigung zu einfach ist.
Wenn sie das Evangelium vorstellen, halten sie es für notwendig, dem Sünder
vorzuhalten, dass er sein Kreuz auf sich nehmen, Jesus nachfolgen und ihn auf
den Thron seines Lebens setzen muss. Einige Prediger glauchen, durch diese
Forderungen Menschen von falschen Bekenntnissen abhalten zu können. Die Antwort
auf unrechte Bekenntnisse liegt jedoch nicht darin, das Evangelium mit der
Bedingung zu verknüpfen, dass der Sünder Nachfolge geloben, Christus gehorchen
und für Ihn leiden soll. Das Evangelium enthätl keine versteckte Klauseln.
Die Antwort für eine echte Bekehrung liegt nicht in diesen Zusätzen, sie liegt
in der richtigen Vorbereitung des Herzens und den Gedanken des Sünders für
das Evangelium. Dies bewirkt der Heilige Geist, wenn der Sünder das Wort Gottes
hört und erkennt, dass er verloren, hilflos und hoffnungslos ist und von Gott,
seinem gerechten, heiligen Schöpfer und Richter verurteilt wird,

\textbf{Vertrauen auf äusserliche, überprüfbare Handlungen}

{[}\ldots{]}

Viele Menschen, die ein `Lippenbekenntnis' abgelegt haben, verlassen sich
darauf, dass sie von Gott angenommen wurden, weil sie einem Aufruf folgten und
nach vorne gegangen sind. Da ein grossteil der evangelistischen Predigt
subjektiv und erfahrungsorientiert ist, wird die Aufmerksamkeit des Hörenden
auf sie selber und ihre persönliche Reaktion auf die Predigt gelenkt.
Begeistert bereichten Christen über die Bekehrung von kleinen Kindern,
Judenglichen und Erwachstenen und setzen es als selbstverständlich voraus,
dass sie das Evangelium auch wirklich verstanden haben und tatsächlich
wiedergeboren sind, nur weil sie eine äusserliche `Entscheidung für Christus'
zur Schau gestellt haben.

{[}\ldots{]}

Wir sollten niemals eine sichtbare Handlung eines öffentlichen Bekenntnisses
als Grundlage dafür anerkennen, dass diejenige Person nun wiedergeboren ist.
Die einzig biblische Grundlage, das Heilsbekenntniss eines Menschen
anzuerkennen, ist sein Verstehen der grundlegenden Wahrheiten des Evangeliums
und sein Vertrauen in dieselben.

{[}\ldots{]}

Unaghängig davon, wie gewissenhaft wir auch im Hinterfragen derer sind, die
eine Bekehrung bekennen, es wird immer solche geben, wie im Gleichnis des
Sämanns, die uns zuerst als Christen erscheinen, aber dann nach einiger
Zeit wieder abfallen. Das Erkennen dieser Gefahr ist ein Beweggrund mehr,
warum wir alles uns mögliche versuchen sollten, die Reinheit, Einfachheit und
Sachlichkeit der frohen Botschaft zu bewahren, damit die Menschen sich auf
die Gerechtigkeit Christi und nicht auf ihre eigene verlassen.

  \bibliography{book.bib,packages.bib}

\printindex

\end{document}
